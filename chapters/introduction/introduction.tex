\label{ref:chapter_1}

Collaboration plays a crucial role in scientific innovation because the problems faced are of such complexity that no single individual or collective has all the relevant information, knowledge or resources \citep{loshin}. Scientific research depends on an efficient and organised means of technological communication so that research can be exploited and improved to its full potential \citep{fair}. This requires special consideration of data resulting from investigation, regardless of whether they are large or small.
  
The better an organisation understands its data, the better it can use it and consequently share it with the scientific community and allow the same data to be repeated and improved. Hence, we can effectively say that the speed at which science evolves today depends on how and whether data sharing is done \citep{10}.

While at first glance it may seem obvious that all research collectives would like to make global collaboration a widespread practice, since we are talking about the evolution of science, this is not necessarily the case \citep{tenopir}. In essence, science is also an industry, and the investment of time and money is something that has to be put on the table and made as efficient as possible \citep{3}. As such, instruments to implement information sharing have to meet certain criteria for research institutions to invest in them.