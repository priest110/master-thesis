Firstly, the objective of this dissertation is to understand how some of the existing software systems and international cooperation networks work to solve the problem mentioned, and, more than understanding how they work, to understand, concretely, what they offer.

With this in mind, once the evaluation is done, the objective is to adapt and apply an existing system so that it can correspond in the most efficient way to scientific research, with special attention to the particularities promoted by the case study of this dissertation, the \gls{sail} project.

Finally, it will be necessary to evaluate the performance of the developed system, important to understand how it might behave in a real context. For this, ideally we would use \gls{macc}, which, by promoting and supporting open scientific initiatives on advanced computing, data science and visualisation, could play a fundamental role in this project in terms of understanding how to adopt a system to real scenarios of considerable computing power. However, in the event that it is not possible for us to use \gls{macc}'s resources, since the installation of Deucalion may still take some time, we will use the resources of \gls{inesctec}, which, although it does not allow us to have the understanding that \gls{macc} would make possible, serves to give us a notion of how the system we develop works and will allow us to extrapolate its use, after the writing of this dissertation, in an infrastructure like \gls{macc}.