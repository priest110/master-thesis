\label{ref:chapter_5}

This final chapter summarizes the main contributions of this dissertation and discusses guidelines for future work.

The work reported in this dissertation addressed the viability of using \gls{sdms}s in the daily life of scientific communities, proposing \textit{\gls{ckan}} as the best solution at the moment. However, we cannot deny that  \textit{\gls{eudat}} is an extremely interesting system and that if one day it is made available for local development it may become a more attractive solution than  \textit{\gls{ckan}}.

The solution produced led to the conclusion that it will be difficult to achieve a universal system that can meet the needs of all scientific institutions or individual researchers. Moreover, the \gls{sdms}s themselves do not appear to be seeking convergence towards universal use \citep{assante}, but we should emphasize that \textit{\gls{eudat}} has the greatest potential to achieve this. As far as the \gls{sail} project is concerned, we are confident that the system based on  \textit{\gls{ckan}} has been able to meet its requirements, in general.

The benchmarking performed showed promising results, and taking into account that our estimation is pessimistic, we believe that the system could respond even better in a real context. Nevertheless, it should be kept in mind that due to the hardware available for testing, we can only extrapolate its usability.

The final conclusion is that the future of research will increasingly depend on these systems, and on how, above all, they will manage to convince the entire scientific community to come on board \citep{poschen}. This will require robust documentation of the systems and developer communities ready to help \citep{vardigan}, as the biggest barrier remains a certain need for technical expertise to exploit the full potential of the technologies \citep{9}.

\section{Future work}

One of the major goals of this dissertation was to recreate the system in an \gls{hpc} environment with the help of an infrastructure such as \gls{macc}. As such, it would be worth exploring this possibility while keeping in mind the guidelines set out. In particular, it would also be interesting to consider for the \gls{macc} other hypotheses beyond the produced system, such as the possible insertion of Deucalion in the  \textit{\gls{eudat}} network,  which would allow customisation and full control of its services.

 \textit{\gls{ckan}} is very close to presenting its new version 2.10. Although we don't know to what extent it may affect the functionality of this system, we have the expectation that it may bring more quality and potential to its use. In addition, it will allow the stability of certain extensions, which is not possible with the current (not stable) master version. In the future, there are two main features that we suggest to be integrated into the system that are not planned for the new version: upload/delete of multiple files and statistics regarding the use of the datasets, such as qualitative evaluation and number of file downloads.

Finally, the strongest suggestion for future work is to turn the software that has been developed into production and explore its functionality in a real-world context. Ideally, it would be interesting to use the system in the service of a research institution like \gls{inesctec} or HASLab.