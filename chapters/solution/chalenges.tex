There were several obstacles encountered during the development of the solution, and some have already been addressed throughout the chapter, but we opted to dedicate this section to explore in more detail those in which we invested more time or which had more consequences on the final result.

The first challenge occurred in installing \textit{\gls{ckan}}, since, on the one hand, the latest stable version of it installed via \textit{Docker} uses \textit{Python} 2, which is no longer supported, while, on the other hand, the master version is in transition to \textit{\textit{\gls{ckan}}} 2.10. We ended up choosing the second option because, after contacting the \textit{\gls{ckan}} team, they informed us that they believed the master version was stable enough, and, in addition, had the advantage of working with \textit{Python} 3.
  
As far as extensions are concerned, there were several that did not work, since they are outdated (ckanext-restricted, ckanext-dataspatial, ckankext-ord-hierarchy, ckanext-oaipmh), or have limited functionality (ckanext-b2find), and therefore could not be used. However, although they added quality to the software they were not central to what we wanted to develop, and we believe that the most important extensions are present in the system.
 
Finally, the biggest obstacle was not being able to take advantage of the partnership with \gls{macc}, for reasons unrelated to the research. For this reason, it was not possible to recreate the best scenario regarding an \gls{hpc} environment. As such, we resorted to using machines from \gls{inesctec} to evaluate the performance of the developed system, which did not allow us to recreate an effective approximation to the real world, as would happen with the \gls{macc} resources.