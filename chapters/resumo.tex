Universalmente, sabe-se que os dados estão em constante mudança, surgem novos conceitos e formas de os representar, e por vezes muitos destes conceitos são depreciados. No contexto da investigação científica, a gestão de dados não é excepção, pois torna-se mais desafiante lidar com o volume crescente de dados e com a complexidade das técnicas que os rodeiam. Além disso, é comum nas empresas, bem como na investigação científica, uma enorme dependência daqueles que são responsáveis pela gestão de toda a informação, levando à possibilidade de as mudanças de equipa não serem tão eficientes como seria de esperar. É, por conseguinte, crucial analisar, organizar e documentar os dados.
    
Há várias técnicas desenvolvidas nos últimos anos para lidar com este problema, e o foco principal desta dissertação é adaptar e aplicar o mais apelativo na análise dos sistemas actualmente conhecidos para lidar com a gestão de dados científicos. Estes sistemas serão comparados considerando os seguintes critérios: arquitectura, interoperabilidade, metadados, usabilidade e segurança. Finalmente, após a adaptação de uma das soluções, será necessária uma avaliação de desempenho a fim de extrapolar o seu funcionamento num contexto do mundo real.

\paragraph{Palavras-chave} Gestão de dados científicos, investigação, CKAN, EUDAT