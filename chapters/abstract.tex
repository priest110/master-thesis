Universally, it is known that data is constantly changing, new concepts emerge and ways to represent them, and sometimes many of these concepts become deprecated. In the context of scientific research, data management is no exception, as it becomes more challenging to deal with the growing volume of data and with the complexity of the surrounding techniques. Furthermore, it is common in companies, as well as in scientific research, a huge dependence on those who are responsible for the management of all the information, leading to the possibility that team changes are not as efficient as one would expect. It is therefore crucial to analyse, organise and document data.
    
There are several techniques developed over the last years to deal with this problem, and the main focus of this dissertation is to adapt and apply the most appealing in the scrutiny of the currently known systems to handle the management of scientific data. These systems will be compared considering the following criteria: architecture, interoperability, metadata, usability and security. Finally, after adapting one of the solutions, a performance evaluation will be required in order to extrapolate its functioning in a real-world context.

\paragraph{Keywords} Scientific data management, research, CKAN, EUDAT